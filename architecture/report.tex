\section{Web Client}

This is the browser which is responsible for a few things.

\begin{itemize}
\item
  The front end code (CSS and HTML) to be displayed
\item
  Every few seconds (3 is a good number) if the program text has been
  changed, then we trigger a save api call to the web server. This will
  save the program and avoid program loss (switching tab, hitting the
  back button, \ldots{}).
\end{itemize}

\section{Local Client}

This is a C library that we will provide to communicate with the
webserver. The idea is that we want people to be able to develop their
code locally if they choose to. Submission for grading is still done on
our server.

\subsection{Components}

\subsection{enforced limitations}

\begin{itemize}
\item
  The sanbox is not enforced in the local client, but it is enforced
  when we perform the grading on the server
\end{itemize}

\section{WebServer}

The webserver is responsible for interacting between the user and the
worker node. Aside from the local client, it is the only public
interface we expose to working on the labs.

The webserver does not require special hardware (does not require a
gpu). It can be hosted on a standard AWS instance and the database
server can be hosted on the same machine.

\subsection{enforced limitations}

\begin{itemize}
\item
  a 3 second limiter is enforced between submissions
\end{itemize}

\subsection{API}

\begin{itemize}
\item
  \textbf{GET} \texttt{/login} : gets the login page
\item
  \textbf{POST} \texttt{/login} (user,pass): checks if the user and a
  SHA hash of the password exists in the users database
\item
  \textbf{GET} \texttt{/signup} : gets the signup page
\item
  \textbf{POST} \texttt{/signup} (user,pass) : creates a new user in the
  database with a SHA hash of the password
\item
  \textbf{POST} \texttt{/logout} : logs out the user (clears the cookie)
\item
  \textbf{GET} \texttt{/oauth\_login} : implements oauth authentication
  (this is used for coursera)
\item
  \texttt{/oauth\_login/callback} : callback for oauth (can be passed
  parameters like the origin page)
\item
  \textbf{GET} \texttt{/grades} : shows the grade page (a table of the
  summary of the grades)
\item
  \textbf{GET} \texttt{/grade/\{mp\_id\}} : shows a detailed view of the
  grade
\item
  \textbf{GET} \texttt{/mp/\{mp\_id\}} : shows the MP interface
\item
  \textbf{POST} \texttt{/mp/\{mp\_id\}/submit} : triggered when one
  clicks the submit button. The programs is passed in the body of the
  \textbf{POST}
\item
  \textbf{POST} \texttt{/mp/\{mp\_id\}/save} : saves the program into
  the database. The programs is passed in the body of the \textbf{POST}
\item
  \textbf{GET} \texttt{/mp/\{mp\_id\}/program} : gets the latest saved
  program for the database. If not found, then the template program is
  read from the file on the server
\item
  \textbf{GET} \texttt{/mp/\{mp\_id\}/description} : gets the
  description of the MP from disk and renders it in markdown format
\item
  \textbf{GET} \texttt{/mp/\{mp\_id\}/attempts} : returns a JSON object
  with a summary of the attempts. The summary contains information such
  as when the attempt was made, the id of the attempt (not shown in the
  table), whether the attempt passed, and which dataset was the attempt
  made againast.
\item
  \textbf{GET} \texttt{/mp/\{mp\_id\}/attempt/\{attempt\_id\}} : gets
  detailed information about the attempt. This includes the same
  information of the attempt summary + program/compiler output + program
  text
\item
  \textbf{POST} \texttt{/mp/\{mp\_id\}/submit\_grade} : this will be a
  button that the user presses to submit the grade to the coursera
  server
\item
  \textbf{POST} \texttt{/mp/\{mp\_id\}/grade} : this will cause the
  server to test the program against all the datasets and provide the
  user with the grade they would get if they submit the program. the
  \textbf{POST} field must contain an API key that is secret and unique
  to each user (could be something as simple as hash(user\_name +
  ``secret''){[}-10:{]})
\item
  \textbf{POST} \texttt{/mp/\{mp\_id\}/run\_results} : this is called by
  the worker. The possible behaviors are: the program failed to compile,
  the program compiled but failed to run, or the program compiled and
  ran. This data is stored into the database as an attempt by the user.
\item
  \textbf{GET} \texttt{/workers} : list all the worker nodes
\item
  \textbf{POST} \texttt{/worker/register} : registers the worker with
  the server . This inserts the worker into the queue.
\item
  \textbf{GET} \texttt{/worker/\{id\}} : list all information about a
  worker node (including log information)
\item
  \textbf{GET} \texttt{/images} : public images are posted here
\item
  \textbf{GET} \texttt{/whoami} : gives the user id of a user. A user
  may need to visit this page to also get information to configure that
  local client (API key, user id, \ldots{})
\item
  \textbf{GET} \texttt{/log} : log output from the server
\item
  \textbf{GET} \texttt{/404} : not found
\item
  \textbf{GET} \texttt{/500} : error
\end{itemize}

\section{Worker}

Unlike the server, the worker must be hosted on a machine with a GPU.
The sandboxing might also enforce a certain OS and a certain revision of
the kernel (the previous implementation was hosted on an ubuntu machine
since centos did not have the kernel features needed for the
sandboxing).

\subsection{enforced limitations}

The server enforces the following limitations:

\begin{itemize}
\item
  no compilation can take more than 10 seconds
\item
  no execution of a program can take more than 3 seconds
\item
  before compilation, the program will be textually checked for black
  listed keywords (exit, abort, include, \ldots{})
\item
  on run, a program will use seccomp and a white list of syscalls. the
  program terminates if a syscall not in the ones approved is used
\end{itemize}

\subsection{API}

\begin{itemize}
\item
  \textbf{POST} \texttt{/mp/\{mp\_id\}/compile} : compiles and runs
  \textbar{} if dataset id is set to -1, then all the datasets are run
  against the program \textbar{}
\item
  \textbf{GET} \texttt{/mp/\{mp\_id\}/data} : checks if the worker has
  the configuration and data required for the MP
\item
  \textbf{POST} \texttt{/mp/\{mp\_id\}/data} : sends the data and
  configuration for the MP to the worker node
\item
  \textbf{GET} /is\_gpu\_alive : runs a simple GPU kernel and checks if
  it terminates without errors. If not, then the gpu is not alive. It is
  the job of the webserver to deal with this (send an email, send
  another command to the worker, remove the worker from the queue,
  \ldots{})
\item
  \textbf{GET} \texttt{/status} : gets the status of the server. This
  includes information such as the current load, disk information,
  \ldots{}
\item
  \textbf{POST} \texttt{/stop} : terminates the server . It is the job
  of the webserver to remove the worker from its queue.
\item
  \textbf{GET} \texttt{/log} : log output from the worker
\end{itemize}

\section{MP Structure}

\subsection{Program Template}

\subsection{Configuration}

\subsection{Dataset}

For each MP

\subsection{Delivery}

\section{Previous Infrastructure}

The previous infrastructure (wb1) is very similar to the one described
above. There are a few omissions however:

\begin{itemize}
\item
  we did not have an instantaneous way to grade MPs. Instead we opted to
  do it off-line
\item
  we did not have a way to check if a GPU worker is down. This caused
  the server to go down for some time without notice (also cause
  misdiagnosis for many users about their bugs and distrust in the
  stability of the server --- more aparent towards the end of the
  course)
\item
  we did not have a local client that people (who do not want to depend
  on our webserver) can develop with
\end{itemize}

\subsection{Implementation}

The previous implementation (wb1) used node.js

\section{New Infrastructure}

\subsection{Why Go?}

\subsection{Why not Haskell?}

\subsection{Why not Python?}

\subsection{Why not Ruby?}

\subsection{Why not Java?}

\subsection{Why not JavaScript?}
